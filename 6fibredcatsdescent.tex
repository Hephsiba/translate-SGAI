\setcounter{chapter}{5}

\chapter{Fibred Categories and Descent}

\setcounter{section}{-1}

\section{Introduction}

Contrary to what was said in the introduction to the previous chapter, it was impossible to make the descent in the category of preschemes, even in special cases, without having developed the language of descent in the general categories beforehand with sufficient care.

The notion of ``descent'' provides the general framework for all processes of ``re-gluing'' objects, and therefore of ``gluing'' of categories. The most classic case of re-gluing relates to the data of a topological space $X$ and of a cover of $X$ by open sets $X_i$; if we consider for all $i$ a space fibre (say) $E_i$ above $X_i$, and for every pair $(i, j)$ an isomorphism $f_{ji}$ of $E_i| X_{ij}$ on $E_j | X_{ij}$ (where we put $X_{ij} = X_i \cap X_j$), satisfying a well-known condition of transitivity (which we write in a short way $f_{kj} f_{ji} = f_{ki}$ ), we know that there is a fibrous space $E$ on $X$, defined as isomorphism with the condition that we have isomorphisms $f_i \maps E | X_i \xrightarrow{\simeq} E_i$, satisfying the relations $f_{ji} = f_jf_i\inv$ (with the usual writing abuse). Let $X'$ be the sum space of $X_i$, which is therefore a fibrous space above $X$ (i.e.\ provided with a continuous mapping $X' \to X$). The data of $E_i$ can be interpreted in a more precise way as a fibrous space $E'$ on $X'$, and the data of $f_{ji}$ as an isomorphism between the two inverse images (by the two canonical projections) $E_1''$ and $E_2''$ of $E'$ on $X''= X' \times_X X'$, the collation condition can then be written as an identity between isomorphisms of fibrous spaces $E_1'''$ and $E_3'''$ on the triple fiber product $X'''= X' \times_X X' \times_X X'$ (where $E_i'''$ designates the inverse image of $E'$ on $X'''$ by the canonical projection of index $i$). The construction of $E$, from $E '$ and $f$, is a typical case of ``descent". Moreover, starting from a fibered space $E$ on $X$, one says that $X$ is ``locally trivial", of fiber $F$, if there exists an open covering $(X_i)$ of $X$ such that the $E| X_i$ are isomorphic to $F \times X_i$, or what amounts to the same, such as the inverse image $E'$ of $E$ on $X' = \coprod_i Xi$ is isomorphic to $X '\times F$.

Thus, the notion of ``gluing'' objects like the ``location'' of a property, are related to the study of certain types of ``basic changes'' $X '\to X$. In algebraic geometry, many other types of basic change, and in particular morphisms $X '\to X$ Thus, the notion of ``gluing'' objects like that of ``localization'' of a property and, are related to the study of certain types of ``basic changes'' $X' \to X$. In algebraic geometry, many other types of change of base, including the flattened $X '\to X$ morphisms must be considered as corresponding to a ``localization'' process with respect to the preschemes, or other objects, ``above'' $X$. This type of localization is used just as much as the simple topological localization (which is a special case besides). The same is true (to a lesser extent) in Analytical Geometry. 

Most demonstrations, reducing to verifications, are omitted or simply sketched out; in this case we specify the less obvious diagrams that are introduced into a demonstration.

\section{Universe, Categories, Equivalence of Categories}

\section{Categories over another}

\section{Change of base in categories over E}

\section{Fibred categories; Equivalence of E-categories}

\section{Cartesian morphisms, inverse images, Cartesian functors}

\section{Fibred categories and prefibred categories. Products and change of base in icelles(?)}

\section{Cloven categories over E}

\section{Cloven Fibration defined from a pseudofunctor}

Call, to abbreviate, pseudofunctor of $\E^{op}$ in $\Cat$ (it should be said, pseudo-normalized functor), a set of data a), b), c) as above, satisfying the conditions A ') and B). 
In the previous section, we have associated a pseudofunctor $\E^{op} \to \Cat$ with a split fibration normalized on $\E$, here we will indicate the inverse construct. 
We will leave the reader to verify most of the details, as well as the fact that these constructions are ``inverse'' to each other. 
Specifically, we should consider the pseudo-functors $\E^{op} \to \Cat$ as the objects of a new category, and show that our constructions provide equivalences, quasi-inverses one from the other, between the latter and the category of cleaved categories above $\E$, defined in the preceding section.

We put $\F_\circ = B$ together sum of sets $Ob(\F (S))$ (N.B. we will note here $\F(S)$ and not $\F_S$ the value in the object $S$ of $\E$ of the pseudofunctor gives, to avoid confusion of notation thereafter). So we have an obvious mapping: 
\[p_\circ \maps \F_\circ \to Ob \E\]. 
Let 
\[\overline\zeta = (S, \zeta), \, \overline \eta = (T, \eta) \qquad (\text{where } \zeta \in Ob\F(S), \, \eta \in Ob\F(T))\]
