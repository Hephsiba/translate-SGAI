\setcounter{chapter}{0}
\chapter{Étale morphisms}

To simplify the exposition we assume that all preschemes in the following are locally Noetherian (at least, starting from section 2).

\section{Basics of differential calculus}

Let $X$ be a prescheme on $Y$, and $\Delta_{X/Y}$ the diagonal morphism $X\to X\times_Y X$.
This is an immersion, and thus a closed immersion of $X$ into an open subset $V$ of $X\times_Y X$.
Let $\ideal_X$ be the ideal of the closed sub-prescheme corresponding to the diagonal in $V$ (N.B. if one really wishes to do things intrinsically, without assuming that $X$ is separated over $Y$ — a \unsure{misleading} hypothesis — then one should consider the set-theoretic inverse image of $\O_{X\times X}$ in $X$ and denote by $\ideal_X$ the augmentation ideal in the above\ldots).
The sheaf $\ideal_X/\ideal_X^2$ can be thought of as a quasi-coherent sheaf on $X$, which we denote by $\Omega_{X/Y}^1$.
It is of finite type if $X\to Y$ is of finite type.
It behaves well with respect to a change of base $Y'\to Y$.
We also introduce the sheaves $\O_{X\times_Y X}/\ideal_X^{n+1}=\mathscr{P}^n_{X/Y}$, which are sheaves of \emph{rings} on $X$, giving us preschemes denoted by $\Delta_{X/Y}^n$ and called the \emph{$n$-th infinitesimal neighbourhood of $X/Y$}.
The polysyllogism is entirely trivial, even if rather long\footnote{cf. EGA~IV~16.3.}; it seems wise to not discuss it until we use it for something helpful, with smooth morphisms.

\section{Quasi-finite morphisms}
